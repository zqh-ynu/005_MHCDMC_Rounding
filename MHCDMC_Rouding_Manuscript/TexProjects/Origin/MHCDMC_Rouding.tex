% weidong81, manuscript 1, 2014

% IEEE/ACM CCGrid 2015
% The Due Date is Jan 30, 2015
\documentclass[11pt,a4paper]{article}

%Package used in the paper
\usepackage{amssymb, amsmath, amsthm}
\usepackage{graphics, color}
\usepackage{multirow}
\def\q{\hfill\rule{0.5ex}{1.3ex}}
\newtheorem{theorem}{Theorem}[section]
\newtheorem{algorithm}[theorem]{Algorithm}
\newtheorem{assumption}[theorem]{Assumption}
\newtheorem{claim}[theorem]{Claim}
\newtheorem{corollary}[theorem]{Corollary}
\newtheorem{definition}[theorem]{Definition}
\newtheorem{lemma}[theorem]{Lemma}
\newtheorem{operation}{Operation}[section]
\newtheorem{proposition}[theorem]{Proposition}
\newtheorem{remark}[theorem]{Remark}
\newtheorem{conjecture}[section]{Conjecture}
\newtheorem{observation}[theorem]{Observation}


\textwidth 16cm \textheight 22.1cm \hoffset=-1.4cm \voffset=-1.6cm
%\textwidth 16cm \textheight 22.7cm \hoffset=-1.6cm \voffset=-2.5cm
%\renewcommand\baselinestretch{1.5} %This sentence is for "double-spaced".

\linespread{1.5}% The \linespread{spacing} command allows you to set any line spacing you like.



\baselineskip=7mm
\title{MHCDMC Rouding}

\author{Qinghui Zhang$^1$, Weidong Li$^2$, Qian Su$^1$ Xuejie Zhang$^{1,}$ \thanks{Correspondence: xjzhang@ynu.edu.cn (X. Zhang)}\\
{\small 1. School of Information Science and Engineering, Yunnan University, Kunming 650504,  China}\\
{\small 2. School of Mathematics and Statistics,  Yunnan University, Kunming 650504, China}}
% \date{}

\begin{document}

\maketitle


\begin{abstract}
abstract

{\bf keywords}: high performance computing; resource allocation;
scheduling; approximation algorithm; bag-of-tasks
\end{abstract}
\newpage
\section{Introduction}
introduction

 \section{Related work}
 related work
 \section{Online scheduling model}
Assume that a heterogeneous  computing system consists of $m$
heterogeneous machines and $n$ users.  Each user $i$ submit  a set
of $a_i$ independent tasks known as a bag-of-tasks \cite{Braun01}
and its profit $p_i$.  As frequently used in scheduling problems for
heterogeneous computing systems \cite{Tarplee151}, let
$\textbf{ETC}=(ETC_{ij})$ be a $n\times m$ matrix where $ETC_{ij}$
is the {\it estimated time to compute} for a task of user $i$ on
machine $j$. Similarly, let $\textbf{APC}=(APC_{ij})$ be a $n\times
m$ matrix where $APC_{ij}$ is the {\it average power consumption}
for a task of user $i$ on machine $j$. Let $x_{ij}$ be the number of
tasks of user $i$ assigned to machine $j$, where $x_{ij}$ is the
primary decision variable in the optimization problem. For a
feasible solution  ${\mathbf x }=(x_{ij})$, the load ( or finishing
time ) of machine $j$ is defined as
 \begin{eqnarray}
L_{j}=\sum_{i=1}^{n}ETC_{ij}x_{ij}.
\end{eqnarray}
It implies that the maximum finishing time of all machines (i.e.,
\emph{makespan}), denoted by $MS({\mathbf x })$, is
 \begin{eqnarray}
MS({\mathbf x })=\max_j L_{j}.
\end{eqnarray}
Correspondingly, the energy consumed by $n$ users is given by:


Let $c$ be the cost per unit of energy. Motivated by the offline
model in \cite{Tarplee14}, we consider the {\it Energy-Aware Profit
Maximizing} (EAPM, for short) problem with bag-of-tasks, which can
be formulated as the following nonlinear integer program (NLIP, for
short).
 \begin{eqnarray}
 Maximize_{\mathbf{x}}   &&  \frac{\sum_{i=1}^np_i-cE({\mathbf x })}{MS(\mathbf{x})}\\
\text{subject to:}&&    \sum_{j=1}^{m}x_{ij}=a_i, \forall i=1,2,\ldots,n \\
 &&   \sum_{i=1}^{n}x_{ij}ETC_{ij}\leq MS(\mathbf{x}),  \forall j=1,2,\ldots,m \\
 &    & x_{ij}\in  \mathbb{Z}_{\geq 0},  \forall i, j.
\end{eqnarray}
The objective of (4) is to maximize the profit per unit time, where
$\mathbf{x}$ is the primary decision variable. The first constraint
ensures that each task in the bag is assigned to some machine.
Because the objective is to maximize the profit per unit time, which
is equivalent to minimize makespan, the second constrain ensures
that $MS(\mathbf{x})$ is equal to the maximum finishing time of all
machines.


However, in practice, when a user arrives, we have to assign all the
 tasks to machines as we do not know the information of the uncoming users.
 Thus, it is necessary to study  the online EAPM problem with bag-of-tasks,
  where the tasks of user $i$ have to assigned
before the  user $i+1$ arrives, for $i=1,2,\ldots,n-1$. Without loss
of generality, assume each task of user $i$ must be assigned to some
machine before the tasks of user $i+1$ arrive, for $i=1$,
$\ldots,n-1$.  Most importantly,  the number of tasks of user $i$ is
very large, we can not assign the $a_i$ tasks one by one. It
motivates us to design an efficient algorithm for the online EAPM
problem with bag-of-tasks.



 \section{An online algorithm}
In this section, we present an efficient algorithm for the online
EAPM problem with bag-of-tasks. For each $i$,  let $L^i_{j}$ and
$E^{i}$ be the {\it  load} of machine  $j$ and the total energy
consumed  after assigning the tasks of the first $i$ users.
Initially, let $L^0_j=0$ for $j=1,\ldots,M$ and $E^{0}=0$. By
definitions, for $i=1,2,\ldots,n$,  we have
\begin{eqnarray}
 L^i_{j}=\sum_{k=1}^{i}x_{kj}ETC_{kj}, \text{ and } E^i=\sum_{k=1}^{i}\sum_{j=1}^{m}x_{kj} APC_{kj}ETC_{kj}.
\end{eqnarray}
For $i=1,2,\ldots,n$, when user $i$ arrives,  we shall decide
$x_{ij}$ such that $\sum_{j=1}^{m}x_{ij}=a_i$ and the objective
value
\begin{eqnarray}
 \frac{\sum_{k=1}^{i}p_k-cE^{i-1}-c\sum_{j=1}^{m}x_{ij} APC_{ij}ETC_{ij}}{MS^i}
\end{eqnarray}
is maximized, where \begin{eqnarray}  MS^i=\max_j  L^i_{j}, \hspace
{1mm} \text{and} \hspace {1mm}  L^i_{j}= L^{i-1}_{j}+x_{ij}ETC_{ij},
\forall i,j.
\end{eqnarray}

Formally, this problem can be formulated as the following integer
program (IP):

 \begin{eqnarray}
\left\{ \begin{split} &  \text {Maximize}   \hspace{2mm}
\frac{\sum_{k=1}^{i}p_k-cE^{i-1}-c\sum_{j=1}^{m}x_{ij} APC_{ij}ETC_{ij}}{MS^i}\\
   & \sum_{j=1}^{m}x_{ij}=a_i \\
  & L^{i-1}_{j}+x_{ij}ETC_{ij}\leq MS^i \\
     &x_{ij}\in Z^{+}\cup \{ 0\}, j=1,\ldots,m.
\end{split}\right.\end{eqnarray}
Note that inequality in IP(11) is equivalent to
\begin{eqnarray}
x_{ij}\leq \lfloor\frac{MS^i-L^{i-1}_{j}}{ETC_{ij}}\rfloor.
\end{eqnarray}


For convenience, sort the tasks of user $i$ in descending order by
$APC_{ij}ETC_{ij}$. Without loss of generality, assume that
\begin{eqnarray}
APC_{i1}ETC_{i1}\geq APC_{i2}ETC_{i2}\geq \cdots \geq
APC_{im}ETC_{im}.
\end{eqnarray}
Our algorithm is based on the following lemma.

 \vspace{1mm}\noindent {\bf Lemma~1.}  {\em There exists an optimal
solution such that
\begin{eqnarray*}
x_{i1}=\cdots=x_{i(\tau-1)}=0,
 \text{and} \hspace{1mm} x_{ij}=\lfloor\frac{MS^i-L^{i-1}_{j}}{ETC_{ij}}\rfloor,
 j=\tau,\ldots,m,
\end{eqnarray*} for some $\tau\in \{1,\ldots,M\}$. }\\
{\bf Proof}. Assume that we know the value of $MS^i$ in the optimal
solution for IP(11). Thus, the objective function of IP(11) is
equivalent to minimize $\sum_{j=1}^{m}x_{ij} APC_{ij}ETC_{ij}$, as
$L^{i-1}_{j}$ and $ETC_{ij}$ are constants. Obviously, to minimize
$\sum_{j=1}^{m}x_{ij} APC_{ij}ETC_{ij}$, $x_{ij}$ with small value
$APC_{ij}ETC_{ij}$ should be maximized and $x_{ij}$ with large value
$APC_{ij}ETC_{ij}$ should be minimized, subject to the constraints
of IP(11).


 In the optimal solution $(x_{i1}, x_{i2}, \ldots, x_{im})$ for IP(11), consider the machine
with minimum index $\tau_1$ such that $x_{i\tau_1}>0$. If there
exists a machine $\tau_2$ ($\geq \tau_1$) such that
$x_{i\tau_2}<\lfloor\frac{MS^i-L^{i-1}_{\tau_2}}{ETC_{i\tau_2}}\rfloor$,
set
\begin{eqnarray}
x'_{ij}=\left\{ \begin{split} & x_{ij}-1, \text{ if } j=\tau_1  \\
   & x_{ij}+1, \text{ if } j=\tau_2 \\
  & x_{ij}, \text{ if } j\neq \tau_1, \tau_2.
\end{split}\right.\end{eqnarray}
It is easy to verify that $(x'_{i1}, x'_{i2}, \ldots, x'_{im})$ is a
feasible solution for IP(11) whose objective value is no less than
that of $(x_{i1}, x_{i2}, \ldots, x_{im})$, as
$APC_{i\tau_1}ETC_{i\tau_1}\geq APC_{i\tau_2}ETC_{i\tau_2}$ (see
(13)). Repeat the above process, until that
  $x_{ij}=\lfloor\frac{MS^i-L^{i-1}_j}{ETC_{ij}}\rfloor$, for any machine $j$ ($\geq \tau$), where
 $\tau$ is the minimum machine index such that $x_{i\tau}>0$.
It implies that we find an optimal solution such that
\begin{eqnarray*}
x_{i1}=\cdots=x_{i(\tau-1)}=0,
 \text{and} \hspace{1mm} x_{ij}=\lfloor\frac{MS^i-L^{i-1}_j}{ETC_{ij}}\rfloor, \text{ for }
 j=\tau+1,\ldots,m.
\end{eqnarray*}
Thus, the theorem holds. \q


Given the value of $MS^i$ in the optimal solution, for $j=m$, $m-1$,
$\ldots$, 1, assign $\lfloor\frac{MS^i-L^{i-1}_j}{ETC_{ij}}\rfloor$
tasks to machine $j$ until all tasks are assigned. According to
Lemma 1, we find an optimal solution.  Although we do not know the
value of $MS^i$ in the optimal solution, $MS^i$ must be in
$\{L^{i-1}_j+kETC_{ij}|k=1,2,\ldots,a_i, j=1,2,\ldots,m\}$. Trying
all possible values of $MS^i$ (at most $O(ma_i)$), we can find the
optimal value of $MS^i$.

For each $\tau=1,\ldots,m$, we only consider the variables $x_{ij}$
($j=\tau,\ldots,m$), which implies that we schedule tasks of type
$i$ to machines of type $j$ ($j=\tau,\ldots,M$). Solving $x_{ij}$ is
equivalent to solve a solution to the following system of linear
equations:
\begin{eqnarray}
\label{Equations}
 \begin{split}
& \sum_{j=\tau}^{m}x_{ij}=a_i;&\\
 &\hspace{1mm} x_{ij}= \frac{MS^i-L^{i-1}_j}{ETC_{ij}},
 j=\tau,\ldots,M&.
\end{split}\end{eqnarray} Note that there are $m-\tau+2$ equations
and $m-\tau+2$ variables $MS_i$ and $x_{ij}$ ($j=\tau,\ldots,M$).
Thus, this system of linear equations can be solved in polynomial
time. For each $\tau=1,\ldots,m$, we obtain a feasible solution
$x_{ij}$. Comparing the objective values of these $m$ solution, we
can find the best solution. Then, for $j=m$ to $1$, we assign
$\lceil x_{ij}\rceil$ tasks of type $i$ to machines of type $j$,
until all tasks are assigned. It is no hard to verify that the
overall running time is polynomial in
$n$ and $m$.





\textbf{{\sc Algorithm 2} } shows the pseudo-code for the online
algorithm.

\noindent\hrulefill\\
{\sc Algorithm 2} Online assigning the tasks of each type to
machines.\\
1: \textbf{For} $i=1$ to $n$ do\\
2:  \hspace{5mm}  Relabel the indices of tasks such that $APC_{i1}ETC_{i1}\geq \ldots \geq
  APC_{iM}ETC_{iM}$;\\
3: \hspace{5mm}   \textbf{For} $\tau=1$ to $m$ do\\
4:           \hspace{15mm}            Solve \ref{Equations} to find a solution $x^{\tau}_{ij}$;\\
5:  \hspace{15mm}                     Comparing these $M$ solution to find
 the best solution $x_{ij}$ such that\\
6:   \hspace{15mm} $\frac{p-cE^{i-1}-c\sum_{j=1}^{M}x_{ij}
APC_{ij}ETC_{ij}}{MS^i}$
   is maximized.\\
7:  \hspace{5mm} \textbf{End for}\\
8:  \hspace{5mm}  \textbf{For} $j=M$ to $1$ do\\
9: \hspace{15mm} Assign $\lceil x_{ij}\rceil$ tasks of type $i$ to
machines of type $j$, until all tasks are assigned;\\
10: \hspace{15mm} Update the $L_j$ of type $j$;\\
11: \hspace{5mm} \textbf{End for}\\
12:  \textbf{End for}\\
 \vspace*{2mm}
\noindent\hrulefill\\
\section{Experimental Results}

\section{Conclusion and future work}
We feel that the local assignment algorithm in Section will find
application in related areas.




\section*{Acknowledgement}
%We are grateful to the anonymous referees for numerous helpful
%comments and suggestions which helped to improve the presentation of
%our work.
The work is
supported in part by the National Natural Science Foundation of
China [Nos. 61662088, 61762091], the Program
for Excellent Young Talents, Yunnan University, and IRTSTYN.

\begin{thebibliography}{}
\bibitem{Abdi17} S. Abdi, L. PourKarimi, M. Ahmadi, F. Zargari, `` Cost minimization for deadline-constrained
bag-of-tasks applications in federated hybrid clouds, '' {\it Future Generation Computer Systems
},  vol. 71,  pp. 113-128, 2017.

\bibitem{Braun01} T. D. Braun, H. J. Siegel, N. Beck, L. L. Boloni, M. Maheswaran,
A. I. Reuther, J. P. Robertson, M. D. Theys, B. Yao, D. Hensgen, and
R. F. Freund, `` A comparison of eleven static heuristics for
mapping a class of independent tasks onto heterogeneous distributed
computing systems, '' {\it Journal of Parallel and Distributed
Computing},  vol. 61, no. 6, pp. 810-837, 2001.

\bibitem{Champati17} J.P. Champati, B. Liang, ``Semi-online algorithms for computational task offloading with
communication delay, '' {\it IEEE Transactions on Parallel and Distributed Systems },
 vol. 28, no. 4, pp. 1189 - 1201, 2017.



 \bibitem{Dayarathna} M. Dayarathna, Y. Wen, and R. Fa, `` Data center energy consumption modeling: a
 survey, ''  {\it IEEE  Communications Surveys \& Tutorials},   vol. 18, no. 1, pp. 732-794 2016.

\bibitem{Diaz14} C. O. Diaz, J. E. Pecero, and P. Bouvry, `` Scalable, low complexity, and
fast greedy scheduling heuristics for highly heterogeneous
distributed computing systems,'' {\it The Journal of
Supercomputing},  vol.  67, no. 3,  pp. 837-853, 2014.

\bibitem{Ebrahimirad} V. Ebrahimirad, M. Goudarzi, and A. Rajabi, `` Energy-aware scheduling for
precedence-constrained parallel virtual machines in virtualized data
centers, '' {\it Journal of Grid Computing}, vol.  13,  no.  2, pp.
233-253, 2015.

\bibitem{Friese12} R. Friese, T. Brinks, C. Oliver, H. J. Siegel, and A. A.
Maciejewski, ``Analyzing the trade-offs between minimizing makespan
and minimizing energy consumption in a heterogeneous resource
allocation problem, '' in {\it  The Second International Conference
on Advanced Communications and Computation}, pp. 81-89, 2012.

\bibitem{Friese13} R. Friese, B. Khemka, A.A. Maciejewski, H.J. Siegel, G.A.
Koenig, S. Powers, M. Hilton, J. Rambharos, G. Okonski, and S. W.
Poole, ``An analysis framework for investigating the trade-offs
between system performance and energy consumption in a heterogeneous
computing environment'', in {\it IEEE 27th International Parallel
and Distributed Processing Symposium Workshops}, pp. 19-30,  2013.

\bibitem{Friese13b} R. Friese, T. Brinks, C. Oliver, H.J. Siegel, A.A. Maciejewski,
and S. Pasricha, ``A machine-by-machine analysis of a bi-objective
resource allocation problem'', in {\it International Conference on
Parallel and Distributed Processing Technologies and Applications},
pp. 3-9, 2013.

\bibitem{Graham69} R. Graham, ``Bounds on multiprocessing timing anomalies, '' {\it SIAM
Journal on Applied Mathematics},   vol. 17,  no. 2,  pp. 416-429,
1969.



\bibitem{Hameed} A. Hameed, A. Khoshkbarforoushha, R. Ranjan, P.
P.h Jayaraman, J. Kolodziej, P. Balaji, S. Zeadally, Q. M. Malluhi,
N. Tziritas, A. Vishnu, S. U. Khan, and A. Zomaya, ``A survey and
taxonomy on energy efficient resource allocation techniques for
cloud computing systems,'' {\it Computing}, vol. 98, pp. 751-774,
2016.


% \bibitem{Huang11} B. C. Huang and T. Jebara, ``Fast b-matching via sufficient selection
% belief propagation, '' {\it  Journal of Machine Learning Research -
% Proceedings Track},   vol.  15, pp. 361-369, 2011.

% \bibitem{Lee} Y.C. Lee, A. Y. Zomaya, `` Energy efficient utilization of resources
% in cloud computing systems, '' {\it The Journal of Supercomputing},
% vol. 60,   no.  2,  pp. 268-280, 2012.



\bibitem{Hsu14} C. H. Hsu, K. D. Slagter, S. C. Chen, and Y. C. Chung, ``Optimizing energy consumption with task consolidation in
clouds, '' {\it   Information Sciences}, vol. 258, no. 10, pp.
 452-462, 2014.

  \bibitem{Karmarkar} N. Karmarkar, ``A new polynomial-time algorithm for linear
  programming, '' in {\it Proceedings of the sixteenth annual ACM symposium on Theory of
  computing (STOC)}, pp.  302-311, 1984.

\bibitem{Kaur15} T. Kaur, and I. Chana, ``Energy efficiency techniques in cloud computing: a survey and
taxonomy, '' {\it ACM Computing Surveys},  vol. 48, no. 2, Article
No. 22, 2015.

\bibitem{Khemka15} B. Khemka,  R. Friese, S. Pasricha, A. A.
Maciejewski, H. J. Siegel, G. A. Koenig, S.  Powers, M. Hilton, R.
Rambharos, and S. Poole,  ``Utility maximizing dynamic resource
management in an over subscribed energy-constrained heterogeneous
computing system,'' {\it  Sustainable Computing: Informatics and
Systems}, vol.  5, pp. 14-30, 2015.

  \bibitem{Khemka152} B. Khemka, R. Friese, L. D. Briceno, H. J. Siegel, A. A.
Maciejewski, G. A. Koenig, C. Groer, G. Okonski, M. M. Hilton, R.
Rambharos, and S. Poole, `` Utility functions and resource
management in an oversubscribed heterogeneous computing environment,
'' {\it IEEE Transactions on Computers},  vol. 64,   no. 1, pp.
2394-2407, 2015.


  \bibitem{Juarez18} F. Juarez, J. Ejarque, and R. M. Badia, ``Dynamic energy-aware
scheduling for parallel task-based application in cloud computing,''
Future Generation Computer Systems, vol. 78, pp. 257-271, 2018.




% \bibitem{Lenstra90} J.K. Lenstra,  D.B. Shmoys, and E. Tardos,  ``Approximation algorithms for
% scheduling unrelated parallel machines, ''  {\it Mathematical
% Programming},  vol.  46,   no. 3, pp.  259-271,  1990.

 \bibitem{Li19} W. Li, X. Liu, X. Cai, X. Zhang, `` Approximation algorithm for the energy-aware profit maximizing problem in
  heterogeneous computing systems,''   {\it
 Journal of Parallel and Distributed Computing}, vol. 124, 70-77, 2019.

 \bibitem{Sajid17} M. Sajid, Z. Raza, `` Energy-aware stochastic scheduler for batch of precedence-constrained
 jobs on heterogeneous computing system,''   {\it Energy}, vol. 125,   pp.
258-274,  2017.



\bibitem{Mashayekhy}  L. Mashayekhy, M.M. Nejad, D. Grosu, Q. Zhang, and W. Shi,
`` Energy-aware scheduling of mapreduce jobs for big data
applications, '' {\it
 IEEE Transactions on  Parallel and Distributed Systems}, vol. 26, no. 10,
pp. 2720-2733, 2015.

\bibitem{Orgerie14} A.-C. Orgerie, M. D. de Assuncao, and L. Lefevre,  ``A survey on techniques
 for improving the energy efficiency of large-scale distributed
 systems, '' {\it ACM Computing Surveys},  vol. 46, no.  4, Article No. 47,  2014.

 \bibitem{Oxley15} M.A.  Oxley, S. Pasricha,  A.A.  Maciejewski, H.J. Siegel, J. Apodaca,
D. Young,  L. Briceno,  J. Smith, S. Bahirat, B. Khemka, A. Ramirez,
and Y. Zou, `` Makespan and energy robust stochastic static resource
allocation of bags-of-tasks to a heterogeneous computing system, ''
{\it IEEE Transactions on Parallel and Distributed Systems},  vol.
26, no. 10, pp.  2791-2805, 2015.

%   \bibitem{Shmoys93} D.B. Shmoys, and E. Tardos,  ``An approximation algorithm
% for the generalized  assignment  problem, '' {\it Mathematical
% Programming},  vol. 62,   no. 1-3, pp. 461-474, 1993.


 \bibitem{Tarplee151} K. M. Tarplee,  R. Friese, A. A. Maciejewski, and H. J. Siegel,
``Scalable linear programming based resource allocation for makespan
minimization in heterogeneous computing systems, '' {\it Journal of
Parallel and Distributed Computing},  vol.  84, pp. 76-86, 2015.
% Previous version in K. M. Tarplee, R. Friese, A. A. Maciejewski, and
% H. J. Siegel, `` Efficient and scalable pareto front generation for
% energy and makespan in heterogeneous computing systems,'' in {\it
% Recent Advances in Computational Optimization: Results of the
% Workshop on Computational Optimization}, pp. 161-180, 2015.

\bibitem{Tarplee152} K. M. Tarplee, R. Friese, A. A. Maciejewski, H. J. Siegel, and E.
Chong,`` Energy and makespan tradeoffs in heterogeneous computing
systems using efficient linear programming techniques, '' {\it IEEE
Transactions on Parallel and Distributed Systems}, vol. 27, no. 6,
pp. 1633-1646, 2016.

% Previous version presented in  K.M. Tarplee, R.
% Friese, A.A. Maciejewski, and H.J. Siegel, ``Efficient and scalable
% computation of the energy and makespan pareto front for
% heterogeneous computing systems, '' in {\it Federated Conference on
%��Computer Science and Information Systems, Workshop on Computational
%��Optimization}, pp. 401-408, 2013.

\bibitem{Tarplee14} K.M. Tarplee, A.A. Maciejewski, and H.J. Siegel,
 ``Energy-aware profit maximizing scheduling algorithm for
heterogeneous computing systems, '' in {\it 14th IEEE/ACM
International Symposium on Cluster, Cloud and Grid Computing}, pp.
595-603, 2014.



\bibitem{Tchana16}  A. Tchana,  N. D. Palma,  I. Safieddine, and D.
Hagimont, ``Software consolidation as an efficient energy and cost
saving solution,'' {\it  Future Generation Computer Systems},
vol. 58, pp. 1-12, 2016.


\bibitem{Zhang18}J. Zhang, N. Xie, X. Zhang, W. Li, ``An online auction mechanism for cloud computing resource allocation
 and pricing based on user evaluation and cost,'' {\it  Future Generation Computer Systems },
vol. 89, pp. 286-299, 2018.

 \bibitem{ZhangF18} F. Zhang, J. Cao, K. Li, S.U. Khan, K. Hwang, ``Multi-objective scheduling of many
  tasks in cloud platforms,'' {\it  Future Generation Computer Systems },
vol. 37, pp. 309-320, 2014.

 \bibitem{Zheng18} W. Zheng, Y. Qin, E. Bugingo, D. Zhang, J. Chen, ``Cost optimization for
  deadline-aware scheduling of big-data processing jobs on clouds,'' {\it   Future Generation
  Computer Systems },
vol. 82, pp. 244-255, 2018.



\bibitem{Clp}(2013, March) Coin-or clp. [Online]. Available:  https://projects.coinor. org/Clp

   \bibitem{Benchmark} (2013, May) Intel core i7 3770k power consumption, thermal.
[Online]. Available: http://openbenchmarking.org/result/
1204229-SU-CPUMONITO81



\end{thebibliography}

\end{document}
